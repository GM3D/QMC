\documentclass[uplatex]{jsarticle}
\usepackage{braket}
\usepackage{amsmath}
\begin{document}

Exercise 1.16.

Verify Eq.(1.95), which is:

\begin{equation}
\hat{M}(x_1, x_2) = \frac{1}{\sqrt{2 \pi}}
 \ket{(x_1, x_2)}\bra{(x_1, x_2)}
\label{eq:target}
\end{equation}

where $\hat{M}(x_1, x_2)$ is defined by
$\bra{x_1}\bra{x_2}\hat{U}\ket{d_2}\ket{d_1}$, with
$\hat{U} = exp\left[-i(\hat{X}\hat{P_1} + \hat{P}\hat{P_2})\right]$
and $\ket{(x_1, x_2)}$ is defined by

\begin{equation}
\braket{x \mid (x_1, x2)} = 
\pi^{-1/4} exp\left[ ixx_2 - \frac{1}{2} (x - x_1)^2\right]
\end{equation}

definition of the initial meter states $\ket{d_i}$ is given by 
\begin{equation}
\braket{x_j \mid d_j} = (\pi / 2)^{-1/4} exp (-{x_j}^2)
\end{equation}

This wavefunction is given in the coordinate representation, but it is
much easier to express these states in momentum representation, since
$p_1$ and $p_2$ become c-number in momentum representation.

wavefunction of $\ket{d_j}$ is not symmetric in $x$ and $p$. momentum
representation of $\ket{d_j}$ is given by:
\begin{equation}
\braket{p_j \mid d_j} = (2\pi)^{-1/4} exp(-p^2/4)
\end{equation}

We will express $\ket{d_1}$ in x-representation and $\ket{d_2}$ in
p-representation in the following calculation.

To estimate the effect of $\hat{M}(x_1, x_2)$ on arbitrary system state,
it is sufficient to see how system state base $\ket{x}$ is mapped by
$\hat{M}$.

\begin{eqnarray}
\hat{M}(x_1, x_2)\ket{x} &=& 
\bra{x_1}\bra{x_2}\hat{U}\ket{x}\ket{d_2}\ket{d_1} \nonumber \\
&=& \frac{1}{\pi} \int_{-\infty}^{\infty} {dx_1}^\prime dp_2 
\bra{x_1}\bra{x_2}e^{-i(\hat{P_1}\hat{X} + \hat{P_2}\hat{P})} 
e^{-{x_1^\prime}^2 - p_2^2 / 4}\ket{x}\ket{{x_1}^\prime}\ket{p_2}\\
\end{eqnarray}

Since $e^{-{x_1^\prime}^2 - p_2^2 / 4}$ is just a c-number, we
concentrate on $\bra{x_1}\bra{x_2}e^{-i(\hat{P_1}\hat{X} + \hat{P_2}\hat{P})} 
\ket{x}\ket{{x_1}^\prime}\ket{p_2}$ for now.

Note that 
\begin{equation}
exp(\alpha A + \beta B) 
= exp(\alpha A) exp(\beta B) exp(-\frac{1}{2} [\alpha A, \beta B])
\end{equation}

if $[\alpha A, \beta B]$ commutes with all of $A$, $B$, $\alpha$, and
$\beta$. Thus, 
\begin{eqnarray}
e^{-i(\hat{P_1}\hat{X} + \hat{P_2}\hat{P})} = 
e^{-i\hat{P_1}\hat{X}} e^{-i\hat{P_2}\hat{P}} e^{\frac{i}{2} \hat{P_1}\hat{P_2}} 
\end{eqnarray}

transforms $\ket{x}\ket{{x_1}^\prime}\ket{p_2}$ as shown in the
following;

\begin{eqnarray}
e^{-i\hat{P_1}\hat{X}} e^{-i\hat{P_2}\hat{P}} 
e^{\frac{i}{2}\hat{P_1}\hat{P_2}}\ket{x}\ket{{x_1}^\prime}\ket{p_2}
&=&
e^{-i\hat{P_1}\hat{X}} e^{-i\hat{P_2}\hat{P}}
e^{\frac{i}{2}\hat{P_1} p_2}\ket{x}\ket{{x_1}^\prime}\ket{p_2}
\nonumber \\
&=& 
e^{-i\hat{P_1}\hat{X}} e^{-i\hat{P_2}\hat{P}}
\ket{x}\ket{{x_1}^\prime - \frac{1}{2}p_2}\ket{p_2}
\nonumber \\
&=&
e^{-i\hat{P_1}\hat{X}} e^{-i p_2\hat{P}}
\ket{x}\ket{{x_1}^\prime - \frac{1}{2}p_2}\ket{p_2}
\nonumber \\
&=&
e^{-i\hat{P_1}\hat{X}}
\ket{x + p_2}\ket{{x_1}^\prime - \frac{1}{2}p_2}\ket{p_2}
\nonumber \\
&=&
e^{-i\hat{P_1}(x + p_2)}
\ket{x + p_2}\ket{{x_1}^\prime - \frac{1}{2}p_2}\ket{p_2}
\nonumber \\
&=&
\ket{x + p_2}\ket{{x_1}^\prime - \frac{1}{2}p_2 + (x + p_2)}\ket{p_2}
\nonumber \\
&=&
\ket{x + p_2}\ket{{x_1}^\prime + x + \frac{1}{2}p_2}\ket{p_2}
\end{eqnarray}

Here we used the basic formula $e^{-ix^\prime\hat{P}}\ket{x} =
\ket{x + x^\prime}$ extensively on each component space.

Using this result,

\begin{eqnarray}
\hat{M}(x_1, x_2)\ket{x} &=& \frac{1}{\sqrt{\pi}}\int e^{-{x_1^\prime}^2
 - p_2^2 / 4}\ket{x+p_2}\braket{x_1 \mid {x_1}^\prime + x +
 \frac{p_2}{2}} \braket{x_2 \mid p_2} d{x_1}^\prime dp_2 \nonumber \\
&=&
\frac{1}{\sqrt{\pi}}\int e^{-{x_1^\prime}^2
 - p_2^2 / 4} \frac{e^{i p_2 x_2}}{\sqrt{2\pi}}\ket{x+p_2} \delta(x_1 - {x_1}^\prime - x -
 \frac{p_2}{2}) d{x_1}^\prime dp_2 \nonumber \\
&=&
\frac{1}{\sqrt{2\pi}}\frac{1}{\sqrt{\pi}}\int e^{-(x_1 - x -
\frac{p_2}{2})^2 - p_2^2 / 4 + i p_2 x_2} \ket{x_2 + p_2} dp_2
\end{eqnarray}

Let $x^\prime = x + p_2$, then the above expression reduces to;

\begin{eqnarray}
&\frac{1}{\sqrt{2\pi}}\frac{1}{\sqrt{\pi}}&
\int e^{-(\frac{x^\prime}{2} + \frac{x}{2} - x_1)^2 
- (x^\prime - x)^2 / 4 + i (x^\prime - x) x_2} \ket{x^\prime} dx^\prime
 \nonumber\\
&=&
\frac{1}{\sqrt{2\pi}}\frac{1}{\sqrt{\pi}}
e^{-\frac{x^2}{2} + xx_1 - x_1^2 -ixx_2} \int e^{-\frac{{x^\prime}^2}{2} +
(ix_2 + x_1)x^\prime}\ket{x^\prime}dx^\prime \nonumber \\
&=& 
\frac{1}{\sqrt{2\pi}}\frac{1}{\sqrt{\pi}}
e^{-\frac{1}{2}(x - x_1)^2 - ixx_2} \int e^{-\frac{1}{2}(x^\prime -
x_1)^2 + ix^\prime x} \ket{x^\prime} dx^\prime
\end{eqnarray}

On the other hand, 
\begin{eqnarray}
\ket{(x_1, x_2)}\bra{(x_1, x_2)}\ket{x} &=& 
\pi^{-\frac{1}{4}} e^{-ixx_2 - \frac{1}{2}(x - x_1)^2}\ket{(x1, x2)}
\nonumber \\
&=& \pi^{-\frac{1}{2}} e^{-ixx_2 - \frac{1}{2}(x - x_1)^2}
\int dx^\prime e^{ix^\prime x_2 - \frac{1}{2}(x^\prime - x_1)^2} \ket{x^\prime}
\end{eqnarray}

Comparing above two expressions concludes (\ref{eq:target}).
\end{document}